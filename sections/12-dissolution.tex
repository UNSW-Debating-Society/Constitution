\newpage
\section{Dissolution}

\begin{enumerate}
\item The Dissolution will occur after the following conditions have been met:
  \begin{enumerate}
  \item An Extraordinary General Meeting is petitioned in writing as set out in section \bref{egm_procedure}.
  \item Procedures for notification as set out in section \bref{egm_procedure} are followed, and the reasons for the proposed dissolution are included in the notice for the meeting.
  \item Quorum for the meeting to dissolve the club shall be 20 members or three-quarters of the club membership, whichever is lesser;
    \begin{enumerate}
    \item Executive members do not count for the purpose of Quorum.
    \end{enumerate}
  \item No other business may be conducted at the meeting to dissolve the club;
  \item After the petitioning body has stated its case any opposition must be given the opportunity to reply, with at least 10 minutes set aside for this purpose;
  \item A vote is taken and the motion to dissolve lapses if opposed by 15 or more members of the club;
  \item If the motion to dissolve is carried, the most recent Members must be notified within 5 days.
  \end{enumerate}
\item The dissolution of DebSoc will also start to occur if the club has been financially and administratively inactive for a period of 18 months.
\item On the dissolution of DebSoc, DebSoc is not to distribute assets to members. All assets are to be distributed to an organisation with similar goals or objectives that also prohibits the distribution of assets to members. This organisation may be nominated at the dissolution meeting of DebSoc. If no other legitimate club or organisation is nominated, Arc will begin procedures to recover any property, monies or records belonging to DebSoc which it perceives would be useful to other Arc-affiliated clubs.
\end{enumerate}
