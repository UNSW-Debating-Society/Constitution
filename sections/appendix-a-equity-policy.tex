\renewcommand\thesubsection{}
\subsection{Appendix A - UNSW Debating Society Equity Policy}
\renewcommand\thesubsection{\Alph{subsection}}

\begin{enumerate}
\item Aim of the Equity Officers
  \begin{enumerate}
  \item The aim of this equity policy is that all members of the UNSW Debsoc should at all times be treated with respect and free from discrimination, harassment and unsafe behaviour.
  \end{enumerate}

\item Application
  \begin{enumerate}
  \item This policy applies to all society members and all attendees of society events.
  \begin{enumerate}
    \item It applies during any event run by the UNSW Debating Society, or event at which a member represents the society.
    \item It applies to interactions between members which arise out of their society membership.
  \end{enumerate}
  \item This policy applies in addition to relevant laws, Codes of Conduct and the requirement to follow reasonable instructions from institutional Co-Equity Officers or tournament equity officers.
  \item Complaints will be treated primarily from the aggrieved individual’s perspective. This policy does not exhaustively state the grounds for equity complaints, individuals are encouraged to discuss any concerns with the Co-Equity Officers.
  \item Voluntary intoxication is not a defence to breaches of this policy.
  \end{enumerate}

\item Prohibited Behaviour
  \begin{enumerate}
  \item Discrimination, bullying, harassment and vilification
    \begin{enumerate}
    \item It is prohibited to discriminate against, bully, harass or vilify another individual or group of individuals based on their actual or perceived attributes including age, debating ability, disability, sex, gender identity, infectious disease status, sexual practice or experience, sexual orientation, political affiliation, pregnancy, race (including colour, descent, national or ethnic origin), religion or socio-economic status.
    \item Positive discrimination for disadvantaged groups is allowed, and debating ability may be considered for the purposes of trials.
    \end{enumerate}
  \item Drugs and alcohol
    \begin{enumerate}
    \item It is prohibited to pressure others into doing an activity they are uncomfortable doing, such as drinking alcohol or consuming illicit drugs.
    \item Members representing the society at intervarsity tournaments must not undermine the tournament, the reputation of the UNSW Debating Society, or teammates’ or other participants’ success through irresponsible behaviour.
    \end{enumerate}
  \item Sexual assault, sexual harassment and consent
    \begin{enumerate}
    \item Allegations of sexual assault and/or harassment will be treated extremely seriously.
    \item Individuals should seek positive consent (explicit confirmation of consent) throughout any romantic or sexual encounter. Consent can be revoked at any time, and consent for one act does not constitute consent for any other sexual act.
    \item Sexual activity should not be engaged in where a party is too intoxicated to meaningfully consent.
    \item Individuals should not use power disparities due to factors such as age, social status or positions of authority to pressure another party into doing a particular act.
    \end{enumerate}
  \end{enumerate}

\item Procedures for Complaints
  \begin{enumerate}
  \item All complaints will be treated confidentially, and further action will only be taken with the consent of the aggrieved individual unless the Co-Equity Officers believe that someone is in danger or a criminal offence warranting police involvement has been committed.
  \item Co-Equity Officers must where possible avoid conflicts of interest, such as where a complaint is made against them or a person with whom they have a close personal relationship. The Executive may appoint additional temporary Equity Officers if the Co-Equity Officers have a conflict of interest or are unavailable.
  \item Equity issues/complaints may be raised by:
    \begin{enumerate}
    \item Contacting or speaking with one of the Co-Equity Officers;
    \item Emailing the Co-Equity Officers at \\unswdebsocequity@gmail.com; or
    \item Submitting a complaint anonymously via an online form that will be made available at tournaments, on social media and on the UNSW Debsoc website. However punitive action against the alleged offender will not be taken where a complaint has been received anonymously.
    \end{enumerate}
  \item After an equity complaint has been raised, the Co-Equity Officers may take action including (but not limited to):
    \begin{enumerate}
    \item Discussing the issue and possible courses of action with the complainant and/or aggrieved individual;
    \item Discussing the incident with the alleged offender and other witnesses;
    \item Facilitating a mediation between the alleged offender and aggrieved individual;
    \item Reporting the incident to the Tournament Equity Officers; and
    \item Involving relevant authorities such as the police or university.
    \end{enumerate}
  \item In consultation with the Executive, Co-Equity Officers can take action including:
    \begin{enumerate}
    \item Restricting the alleged offender from attending any/all UNSW Debating Society events for a period of time or indefinitely;
    \item Denying future subsidies to attend tournaments; and
    \item Restricting from trialling for or representing the society at tournaments.
    \end{enumerate}
  \item The alleged offender must be informed of the allegations against them and given the opportunity to respond before punitive action is taken (except where the Co-Equity Officers believe that it is necessary to involve relevant authorities).
  \item Debsoc and the Co-Equity Officers are not bound by the rules of procedural fairness where adherence to those rules would endanger the physical or psychological safety of any person: \label{procedural_fairness}
    \begin{enumerate}
    \item Where section \ref{procedural_fairness} applies, the rules of procedural fairness must be adhered to, but only to the extent that adherence does not endanger the physical or psychological safety of any person.
    \end{enumerate}
  \end{enumerate}
\end{enumerate}
