\newpage
\section{UNSW Debating Society Grievance Resolution Policy}

\begin{enumerate}
\item Purpose and Scope
  \begin{enumerate}
  \item This policy guides the resolution of grievances within UNSW Debsoc.
  \item It applies to all members and attendees of society events.
  \end{enumerate}

\item Principles
  \begin{enumerate}
  \item Fairness, respect, confidentiality and timely resolution guide all grievance processes.
  \item The policy aims to resolve conflicts constructively and maintain a supportive society environment.
  \end{enumerate}

\item Grievance Resolution Procedures
  \begin{enumerate}
  \item Informal Resolution
    \begin{enumerate}
    \item Members are encouraged to resolve grievances informally through discussion or mediation.
    \end{enumerate}
  \item Formal Resolution
    \begin{enumerate}
    \item If informal resolution fails or is inappropriate, a formal written grievance may be submitted to the Executive or Co-Equity Officers.
    \item The Executive or Co-Equity Officers will investigate and take appropriate action.
    \end{enumerate}
  \item External Referral
    \begin{enumerate}
    \item If necessary, grievances may be referred to relevant university or legal authorities.
    \end{enumerate}
  \end{enumerate}

\item Roles and Responsibilities
  \begin{enumerate}
  \item The Executive and Co-Equity Officers are responsible for managing grievance processes.
  \item Members must cooperate with investigations and respect outcomes.
  \item Confidentiality must be maintained except where disclosure is legally required.
  \end{enumerate}

\item Outcome and Follow-Up
  \begin{enumerate}
  \item Appropriate sanctions or remedies will be applied where grievances are substantiated.
  \item The Society will monitor and review grievance trends to improve culture and prevent recurrence.
  \end{enumerate}
\end{enumerate}
