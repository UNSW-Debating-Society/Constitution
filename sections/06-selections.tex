\newpage
\section{Selections} \label{selections_section}

\begin{enumerate}
\item Selections Policy
  \begin{enumerate}
  \item The Vice President (Externals) shall ensure a policy is approved by the Executive regarding the manner of selections for all Intervarsity tournaments. There may be different policies for different tournaments. This policy will be used upon approval by the Executive.
  \item Such policies shall be put into writing and a copy kept with the Secretary/Arc Liaison.
  \item Prior to selections for a tournament, members shall be made aware of the policy pertaining to that tournament and will agree to the policy by signing a copy of the policy.
  \item If a member believes that not all procedures were followed in trials, selections will not be invalidated if they have been conducted duly and fairly. This is firstly to be decided by the panel constituted under \ref{flying_resolution}, and if this is not satisfactory for all parties, then trial results may be ratified or a retrial ordered by a general vote at an AGM or EGM. \label{improper_trial_resolution}
    \begin{enumerate}
    \item An EGM held for the purposes of \ref{improper_trial_resolution} must comply with the requirements of \ref{egm_procedure}.
    \end{enumerate}
  \item There shall be separate selections for each intervarsity tournament.
  \item The selections shall be conducted by independent selectors on the basis of merit, subject to other provisions in section \ref{selections_section}, and any variations should be done with the unanimous agreement of all affected individuals.
  \item Unless otherwise provided by this section \ref{selections_section}, ranked lists are to remain confidential.
  \item An individual is not to be put into the same team as a person against whom they have declared a conflict:
    \begin{enumerate}
    \item The conflict must have been declared to the Co-Equity Officers prior to the release of the teams, unless special circumstances exist which make non-declaration prior to release of the contingent reasonable.
    \end{enumerate}
  \item The selections policy submitted by the VP (Externals) is not to allow for the submission of CVs by debaters to trialists.
  \end{enumerate}

\item Australasian Intervarsity Debating Championships
  \begin{enumerate}
  \item Debaters may indicate preferences for people with whom they would like to be in a team. The number of permitted preferences is 0 or 1.
  \item Selectors will watch every debater at least once.
    \begin{enumerate}
    \item In special circumstances, debaters may trial by video.
    \end{enumerate}
  \item Selectors will select debaters in contention to make the contingent.
    \begin{enumerate}
    \item Selectors may watch selected debaters in at least one callback.
    \item Debaters offered a callback may decline.
    \end{enumerate}
  \item Selectors will produce a ranked list of debaters based on merit.
    \begin{enumerate}
    \item Selectors will form teams based on the ranked list, subject to any affirmative action requirements and conflicts.
    \item In forming teams, selectors may (but are not required to) also consider team preferences for individual debaters provided that:
      \begin{enumerate}
      \item Both individuals preferenced each other;
      \item The lower ranked individual must be within 5 places in ranking of the highest ranked individual in that team; and
      \item The selector believes that forming the team to take into account preferences would not detract from the contingent’s likelihood of success.
      \end{enumerate}
    \item An individual must not move to a higher ranked team as a result of preferencing than they would otherwise have been selected for and cannot be moved into the contingent as a result of preferences if they would not otherwise have been selected.
    \end{enumerate}
  \end{enumerate}

\item World Universities Debating Championships
  \begin{enumerate}
  \item Selectors will watch every debater at least once.
    \begin{enumerate}
    \item In special circumstances, debaters may trial by video.
    \end{enumerate}
  \item Selectors will select debaters in contention to make the contingent.
    \begin{enumerate}
    \item Selectors will watch selected debaters in at least one callback.
    \item In special circumstances, debaters offered a callback may decline.
    \end{enumerate}
  \item Selectors will produce a ranked list of debaters based on merit.
    \begin{enumerate}
    \item Other considerations such as affirmative action are to be applied independently by the Vice President (Externals).
    \item The ranked list will be disclosed to all debaters.
    \item Selectors may provide recommended team combinations in addition to the ranked list.
    \end{enumerate}
  \item The contingent of debaters will be selected in order from the ranked list.
  \item A contingent meeting will be held as soon as practicable.
    \begin{enumerate}
    \item Debaters who are unable to attend may be contacted via other means.
    \end{enumerate}
  \item The first ranked debater may ask any debater from the contingent to be in their team. \label{wudc_contingent_forming_procedure}
    \begin{enumerate}
    \item Any debater may decline.
    \item Any debater who by virtue of the ranked list is skipped over may object.
    \item If objection is raised, the first ranked debater may ask the objecting debater to be in their team or elect to drop down.
    \item If no objections are raised, a team is formed.
    \end{enumerate}
  \item The next ranked debater who is not currently in a team repeats the process outlined in \ref{wudc_contingent_forming_procedure}.
  \item Any team which causes a failure in affirmative action may elect to drop down and remain unaffected by that policy.
  \item When all teams have been formed, the debating contingent has been formed.
    \begin{enumerate}
    \item Any additional debating positions granted will be offered in order to the highest-ranked debaters not already selected in the debating contingent, subject to affirmative action requirements, until those positions are filled.
    \item If any debating position is vacated before two weeks have passed since the debating contingent was first formed, the debating contingent will no longer be considered formed and will be re-formed as per the process outlined in \ref{wudc_contingent_forming_procedure}.
    \item If any debating position is vacated after two weeks have passed since the debating contingent has been formed, the position will be offered in order to the highest-ranked debaters not already selected in the debating contingent, subject to affirmative action requirements, until that position is filled. \label{delayed_wudc_speaker_vacation}
    \item Any variation from \ref{delayed_wudc_speaker_vacation} must have the unanimous consent of all affected individuals.
      \begin{enumerate}
      \item An affected individual, for the purposes of this section, is taken to mean an individual in a team that is being altered under this section, or an individual in a team that is being re-numbered as a result of \ref{delayed_wudc_speaker_vacation}.
      \end{enumerate}
    \end{enumerate}
  \end{enumerate}

\item The Punitive Rule
  \begin{enumerate}
  \item Members who withdraw from a contingent to an intervarsity tournament after the contingent has been selected shall have the punitive rule applied, unless there is a valid reason for their withdrawal.
  \item Valid reasons include, but are not limited to:
    \begin{enumerate}
    \item Illness; and
    \item Death of a family member.
    \end{enumerate}
  \item If a member believes there was a valid reason for their withdrawal, they must submit their reasoning in writing to the Executive. The Executive shall make an assessment on whether there is a valid reason by vote, but must consider:
    \begin{enumerate}
    \item Legally valid and verifiable documentation (e.g. a medical certificate, a death certificate).
    \item Voluntary steps taken by the member to remedy the disruption caused by their withdrawal (e.g. dropping out as early as possible, covering some or all costs, assisting in finding a replacement to take their place in the contingent).
    \item The objective seriousness of the members’ withdrawal from the contingent in terms of disruption caused (e.g. financial liability, inability of other contingent members to attend the tournament).
    \item Obligations under relevant anti-discrimination legislation.
    \item Obligations under relevant Arc policies.
    \end{enumerate}
  \item Punishments under the punitive rule may include, but are not limited to:
    \begin{enumerate}
    \item Increasing service points requirements;
    \item Financial liability for registration fees;
    \item Ineligibility for selection for the next major tournament which they are eligible for, excluding AWGMDC.
    \end{enumerate}
  \item The punishment applied under the punitive rule shall be subject to the discretion of the Executive, but must be proportionate to the objectives of:
    \begin{enumerate}
    \item Indemnifying the society from liabilities caused by the member’s withdrawal.
    \item Deterring future withdrawals that might cause disruption to future contingents.
    \end{enumerate}
  \item In all decisions taken by the Executive under the Punitive Rule, conflicts of interest must be declared and recorded in the minutes:
    \begin{enumerate}
    \item The Executive must call a meeting within 14 days of the member’s withdrawal from the contingent, to be held no later than 21 days from the member’s withdrawal.
      \begin{enumerate}
      \item If no vote takes place within 21 days, the member is deemed exempted from the Punitive Rule.
      \end{enumerate}
    \item The Executive must declare and document conflicts of interest in the minutes.
    \item The Executive must document reasons for the decision, including evidence of due consideration of the factors contained in this section, in the minutes.
    \end{enumerate}
  \end{enumerate}

\item The Participation Rule
  \begin{enumerate}
  \item The Executive may determine whether debaters or adjudicators need to satisfy a participation requirement in order to qualify for selections.
  \item The participation requirement must be one that can reasonably be fulfilled and members must be given adequate notice of the requirement so they have a fair chance to take steps to satisfy it.
  \item Any participation rule must be applied to all members in a consistent way.
  \end{enumerate}

\item Members who owe DebSoc money shall be ineligible for selection to an Intervarsity tournament until the debt is resolved.
  \begin{enumerate}
  \item Unless exempted by the Executive.
  \end{enumerate}

\item Affirmative action requirements of Intervarsity Debating Organisations shall be taken into account in selections.

\item For WUDC, an affirmative action requirement will still apply. This affirmative action requirement is that at least one third of the top three teams and one third of the entire contingent (including adjudicators who fall within adjudicator caps) must be non cis-male. Affirmative action requirements only apply subject to sufficient non cis-males trialling to fill such positions.

\item Novice participation requirements shall be taken into account in selections for the Australian Debating Championships (‘Easters’).

\item At least one half of the debating contingent and at least one half of the entire contingent (including adjudicators who fall within adjudicator caps) and at least one third of the highest ranked debating team for the Australian Intervarsity Debating Championships (‘Easters’) must be non cis-male. \label{easters_ncm_aa}
  \begin{enumerate}
  \item If section \ref{easters_ncm_aa} cannot be satisfied because an insufficient number of non cis-men trial, section \ref{easters_ncm_aa} should be applied until no further non cis-men can be selected in the contingent.
  \end{enumerate}
\end{enumerate}
