\newpage
\section{Executive}
\begin{enumerate}
\item The Executive shall consist of the following positions: \label{exec_positions}
  \begin{enumerate}
  \item President;
  \item Vice-President (Income/Treasurer);
  \item Vice-President (Externals);
  \item Vice-President (Internals);
  \item Vice President (Competitions);
  \item Secretary/Arc Liaison;
  \item Events Director;
  \item WGM Officer;
  \item Development Officer;
  \item First Year Representative;
  \item Immediate Past President;
    \begin{enumerate}
    \item If the Immediate Past President is not a current UNSW student or Arc member, the previous clause will not apply.
    \end{enumerate}
  \item Welfare Officer;
  \item POC Officer;
  \end{enumerate}
\item All positions, with the exception of Immediate Past President and First Year Representative, shall be elected according to the procedure in section 8.
  \begin{enumerate}
  \item The First Year Representative shall be elected from first year members of Debsoc at an Extraordinary General Meeting as soon as practicable after Easters.
    \begin{enumerate}
    \item The First Year Representative must be an ordinary member of Debsoc in their first year of membership of Debsoc.
    \item Only members in their first year of membership of Debsoc are permitted to vote for the First Year Representative.
    \end{enumerate}
  \item The WGM Officer must be non cis-male.
    \begin{enumerate}
    \item Only WGM members of Debsoc are permitted to vote for the WGM Officer.
    \end{enumerate}
  \item The POC Officer must be a person of colour.
    \begin{enumerate}
    \item Only POC members of Debsoc are permitted to vote for the POC Officer.
    \end{enumerate}
  \item The IPP shall be the President of the society from immediately previous year. \label{ipp_desc}
  \end{enumerate}
\item The Executive shall appoint a Co-Equity Officer (non cis-male), Co-Equity Officer (male) and Tab Officer at the first meeting of the Executive:
  \begin{enumerate}
  \item The Co-Equity Officers and Tab Officer do not need to be existing members of the Executive.
  \end{enumerate}
\item The terms of the Executive:
  \begin{enumerate}
  \item All positions shall commence at the close of the AGM where the Executive has been voted in and shall finish at the following AGM, with the exception of the roles of Externals and Treasurer.
  \item The Vice President (Externals) shall be responsible for the external competitions of the calendar year, unless negotiated otherwise.
  \item The term of the Treasurer shall be the calendar year.
  \item While these positions continue beyond the time of the AGM, both of these Executive members should be involved in the transition process that the Executive undertakes to the new appointees.
  \end{enumerate}
\item A casual vacancy occurs for an Executive position if:
  \begin{enumerate}
  \item The person passes away.
  \item The person resigns in writing to the Executive.
  \item The person is not a student of UNSW.
  \item The person goes on exchange for a trimester without full disclosure at the AGM at which the person was elected.
  \item A vote of no confidence is passed at an Extraordinary General Meeting.
  \item The person is absent from any three Executive meetings.
  \item Unless exempted unanimously by the Executive, the person fails to fulfil reasonable obligations delegated by:
    \begin{enumerate}
    \item The Constitution, or
    \item The Executive.
    \end{enumerate}
  \item A new position of the Executive is created outside of the notice period prescribed in \ref{agm_notice_proc} for an Arc-compliant AGM.
  \end{enumerate}
\item Casual vacancies may be filled by either:
  \begin{enumerate}
  \item Temporarily co-opting a member of Debsoc to a position by resolution of the Executive - such a person shall have no voting power and shall serve for a period of no longer than one month, whereby their position will be filled in accordance with \ref{vacancy_egm}; or
  \item Electing someone to the position at an Extraordinary General Meeting. \label{vacancy_egm}
  \end{enumerate}
\item A person who causes a casual vacancy may be subject to punitive measures determined by the Trustees including but not limited to:
  \begin{enumerate}
  \item Ineligibility for selection to an Intervarsity tournament for one calendar year.
  \item Ineligibility for subsidies to an Intervarsity tournament for one calendar year.
  \item Public disavowal of the person and their contribution to DebSoc.
  \end{enumerate}
\item The roles of the executive are:
  \begin{enumerate}
  \item To further the objects of Debsoc;
  \item To formulate general policy for the furtherance of the objects of Debsoc;
  \item To conduct the day to day running of Debsoc; and
  \item To oversee the prudent financial management of Debsoc.
  \end{enumerate}
\item The Executive shall meet:
  \begin{enumerate}
  \item At least once every month for a minimum of 10 months per year; and
  \item When called upon by the President; and
  \item When requested by three members of the Executive; and
  \item In open session unless the Executive resolves to discuss a matter in closed session.
  \end{enumerate}
\item A Valid Executive Resolution shall require:
  \begin{enumerate}
  \item That the matter to be dealt with by resolution be identified as an agenda item when notice of the relevant meeting is sent to executive members, unless by unanimous resolution of present voting executive members, it is declared the matter shall be dealt with.
  \item That at least 50\% of voting members of the executive be present when the resolution is passed.
  \item A simple majority of present voting executive members to vote in favour of the resolution. In the event of a tie, the President shall have the casting vote. Executive members not present at the relevant meeting may nominate in advance another voting Executive member to act as their proxy. Each voting Executive may only carry one proxy.
  \end{enumerate}
\item In the event that an important decision (as specified below) must be taken in the time between Executive meetings, the motion must be passed by means of a flying resolution: \label{flying_resolution}
  \begin{enumerate}
  \item All executive members can request that a decision be put to a flying resolution.
  \item A decision about a flying resolution shall be determined by the President and the executive member who is responsible for the portfolio to which the decision relates. If the decision is a financial one, the VP (Income/Treasurer) must also be consulted.
  \item Decisions that may be taken as flying resolutions may include, but are not limited to:
    \begin{enumerate}
    \item Issues to do with selections and trials for teams and adjudicators for competitions.
    \item Subsidies for teams and adjudicators.
    \item Major expenditures not included in the draft budget.
    \end{enumerate}
  \item The procedure for a flying resolution will involve:
    \begin{enumerate}
    \item An email will be sent to the entire executive committee setting out the decision to be taken.
    \item The following people must vote: at least two of the trustees (ie. President, Vice President (Income/Treasurer), or Vice President (Externals)) and at least three other members of the executive committee. If the decision relates to one or more particular portfolios (as determined by the President), then the holders of those portfolios must vote; in addition to or as part of the three additional executive members. Voting will be either in the form of an email indication support or opposition for the resolution to the exec email account or to the relevant nominated email account.
    \end{enumerate}
  \item For the resolution to pass, there must be majority support amongst all voters and majority support amongst the three trustee members.
  \item In the case that one or more of the Trustee Executive members or the executive member to whose portfolio the decision relates cannot be contacted, or fails to establish contact, the President, shall determine whether or not the flying resolution can be passed in the absence of that or those executive member(s).
  \end{enumerate}
\item Approved minutes of Executive meetings shall be available to members.
\item The Executive:
  \begin{enumerate}
  \item May delegate its authority, except for this power of delegation (unless explicitly resolved). The Executive may create or revoke any delegations by resolution.
  \item The Executive is at all times bound by decisions of an Annual General Meeting or Extraordinary General Meeting of Debsoc.
  \end{enumerate}
\item With the exception of the positions of Co-Equity Officer (male), Co-Equity Officer (female), and Immediate Past President, no person may hold two of the positions outlined in section \ref{exec_positions}.
\item Debsoc must not endorse any candidate or group of candidates in any election, except where a motion to do so has been passed by a two-thirds majority of the Executive. \label{exec_endorsements}
  \begin{enumerate}
  \item Any motion to endorse a candidate or group of candidates must specify the reasons for endorsing them, and those reasons must be noted in the minutes.
  \item For the purposes of section \ref{exec_endorsements}, endorsing a candidate includes any act of publicising or expressing support for a candidate through any medium associated with Debsoc.
  \item Endorsing a candidate does not include any actions taken by members of the Executive in their personal capacity.
  \end{enumerate}
\item The following provisions apply if a member of the Executive has a conflict of interest: \label{coi_provisions}
  \begin{enumerate}
  \item A member of the Executive must inform the Executive if they have a conflict of interest in relation to a matter being voted on or discussed by the Executive.
  \item For the purpose of sections \ref{coi_provisions}, a member of the Executive shall be deemed to have a conflict of interest if they have an interest that is likely to materially affect their vote, or is likely to be perceived by members of the Society as materially affecting their vote.
    \begin{enumerate}
    \item Matters in which a member will have a conflict of interest include, but are not limited to:
      \begin{enumerate}
      \item Matters relating to selections policy which are likely to affect the member’s ranking in those trials.
      \item Matters relating to the endorsement of the member (or a group of which the member is part) in an election.
      \end{enumerate}
    \item Members will not have a conflict of interest in:
      \begin{enumerate}
      \item Matters involving practical arrangements for trials which are unlikely to affect the member’s result in those trials.
      \item Matters involving determination of subsidies pursuant to section \ref{subsidy_determination_deadline}.
      \item Matters involving an allocation of funding to their portfolio.
      \end{enumerate}
    \item Any dispute about whether the member has a conflict of interest shall be resolved by a vote of all Executive members.
    \end{enumerate}
  \item Any action to be taken as a result of an Executive member’s conflict of interest shall be decided by a vote of all Executive members. Such action may include, but is not limited to:
    \begin{enumerate}
    \item Noting the existence of the conflict of interest in the minutes of the meeting.
    \item Preventing the member of the Executive with a conflict of interest from being present while the matter which is the subject of the conflict of interest is being discussed.
    \item Preventing the member of the Executive with a conflict of interest from voting on a motion related to the matter which is the subject of the conflict of interest.
    \end{enumerate}
  \end{enumerate}
\item Constitutional disputes will be settled by a panel of the trustees. If one of the members has a conflict of interest, then they will be excluded. The panel must have at least three members. The decision must be distributed in the next Debsoc newsletter and on the Debsoc website. If such a panel cannot be constructed, an EGM will be called to resolve this at the soonest possible time meeting all the requirements for advance notice.
\end{enumerate}
