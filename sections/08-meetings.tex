\newpage
\section{Meetings}

\begin{enumerate}
\item Annual General Meetings
  \begin{enumerate}
  \item An Annual General Meeting must be held once per year to report on the year and elect the Executive for the next year.
  \item An Annual General Meeting must be called with 14 days’ notice by email to the Members of Debsoc, and shall be held during University session. \label{agm_notice_proc}
  \item At an Annual General Meeting, reports must be received from the President and Treasurer, full financial reports must be received and full elections (except for First Year Representative) for a new Executive must be held.
  \item Any constitutional amendments may be voted on only if 7 days notice is given.
  \item In order to be eligible to vote in an Annual General Meeting, a member must already be a member of the society, as recorded by SpArc, before notice of the election is given.
  \end{enumerate}

\item Election Procedure
  \begin{enumerate}
  \item At least 14 days prior to the AGM date:
    \begin{enumerate}
    \item Notice of the AGM and nomination procedure shall be given to members,
    \item A Returning Officer shall be appointed by the Executive, and
    \item Nominations shall be accepted.
    \end{enumerate}
  \item Members will nominate their candidacy to the Returning Officer in portfolio(s) in order of preference.
  \begin{enumerate}
    \item Candidates must not run in coalitions.
    \item Candidates must not campaign until 5 days prior to the AGM.
  \end{enumerate}
  \item 5 days prior to the AGM, an official ballot shall be sent to members.
    \begin{enumerate}
    \item The order of candidates for each portfolio will be random.
    \item The ballot will disclose the preferred portfolio of each candidate.
    \item Campaigning for election at an Annual General Meeting will take place subject to the following procedures:
      \begin{enumerate}
      \item The outgoing President may not support or criticise any candidate for election.
      \end{enumerate}
    \end{enumerate}
  \item Additional nominations received at the AGM are valid.
  \item Voting is to be conducted at the AGM:
    \begin{enumerate}
    \item Optional preferential voting shall be used.
    \item Voting for all portfolios shall be conducted simultaneously.
    \item If a candidate wins more than one portfolio, the candidate shall be elected to their most preferred portfolio, and votes for the vacated portfolio shall be counted again.
    \end{enumerate}
  \end{enumerate}

\item Extraordinary General Meetings \label{egm_procedure}
  \begin{enumerate}
  \item An Extraordinary General Meeting may be held if the Executive sees fit, or if petitioned under this clause.
  \item An Extraordinary General Meeting must be called with fourteen (14) days notice by email to the Members of Debsoc.
  \item An Extraordinary General Meeting will be held if petitioned by 15 members of Debsoc or half the club's membership, whichever is the lesser. The petition must be lodged with the Executive, who will call the meeting for within 21 academic days, but not within fourteen (14) academic days.
  \item If a motion of no confidence is passed at an Extraordinary General Meeting to an Executive position, a casual vacancy shall occur and will be voted upon immediately according to this Constitution.
  \item Any constitutional amendments may be voted on only if included in the notice of the meeting.
  \end{enumerate}

\item Returning Officer
  \begin{enumerate}
  \item They ensure that they are at all times impartial and objective and cannot be determined to have a real or perceived conflict of interest by Club members, Executive or by Arc Clubs Management.
  \item They ensure that all elections are run fairly and in line with the rules set out by this Club’s Constitution and according to Arc Clubs Policy and Procedure.
  \item They prepare and circulate all notices of election, nominations, voting and proxies to be held as part of any General Meeting in which an election is to take place.
  \item They provide all members with access to an email address that is designated for use by the Returning Officer over the course of their duties.
  \item They accept all nominations submitted that satisfy the rules of this Club’s Constitution and Arc Clubs Policy and treat any defective or late nominations in the manner prescribed by this Club’s Constitution and/or Arc policy.
  \item If voting is to take place online, they ensure that the appointed Returning Officer(s) are the only person(s), alongside Arc Clubs Management, with access to the voting forms and spreadsheets.
  \item They run the portion of the General Meeting pertaining to the election of candidates.
  \item They allow for at least 1 scrutineer per candidate, (who cannot be the candidate themselves) to be present for the counting of votes, if this is held in person, or for that person to be provided access to the voting sheets if the election was held online.
  \end{enumerate}

\item Quorum for Annual General Meetings and Extraordinary General Meetings is 15 members of Debsoc or half of Debsoc's membership, whichever is the lesser.
  \begin{enumerate}
  \item Executive members do not count for the purpose of Quorum.
  \end{enumerate}

\item Votes for elections to the Executive (including a casual vacancy) conducted at Annual General Meetings or Extraordinary General Meetings must be by secret ballot.

\item A minimum of three different members shall remain on the Executive at all times.

\item No more than one person may be elected to a particular Executive position.

\item Proxies are not allowed at General, Annual General or Extraordinary General Meetings.

\item A General Meeting may be held when the Executive sees fit.
\end{enumerate}

