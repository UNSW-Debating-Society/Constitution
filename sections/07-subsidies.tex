\newpage
\section{Subsidies}

\begin{enumerate}
\item To further its objects, DebSoc may give a subsidy to those members participating in a given intervarsity tournament for part or all of the cost of registration and transportation to the tournament.
\item The Executive in a given year may decide:
  \begin{enumerate}
  \item To grant subsidies to participants in some tournaments but not others;
  \item To grant no subsidies at all that year for financial reasons;
  \item To grant different levels of subsidy to some teams going to a tournament over others, including giving no subsidy to some teams;
  \item To grant a higher level of subsidy to novice debaters going to a tournament over experienced debaters, including giving no subsidy to experienced debaters;
  \item To grant a higher level of subsidy to adjudicators going to a tournament over debaters, including giving no subsidy to debaters.
  \end{enumerate}
\item Decisions regarding the value of subsidies must be made by January 31st of the year in which those subsidies will be given. After this date, any decision regarding the value of subsidies may only be made at an AGM or EGM. \label{subsidy_determination_deadline}
\item Decisions on subsidies should take into account:
  \begin{enumerate}
  \item The prestige of tournaments;
  \item The cost of attending tournaments;
  \item The benefit to DebSoc of attending tournaments;
  \item The cost of the proposed subsidy to DebSoc;
  \item The benefit of the proposed subsidy in increasing the size and competitiveness of the contingent to tournaments.
  \end{enumerate}
\item In order to be eligible for a subsidy, persons must:
  \begin{enumerate}
  \item Be a member of DebSoc;
  \item Be representing UNSW at the tournament;
  \item Have fulfilled all relevant service requirements. \label{service_req_exemption}
  \end{enumerate}
\item The service requirements apply to members attending the Australasian Intervarsity Debating Championships and the World Universities Debating Championships. At the start of each year, the executive sets requirements calibrated by a points system in regards to three criteria:
  \begin{enumerate}
  \item Debating with novice members;
  \item Adjudicating members;
  \item Serving the society.
  \end{enumerate}
\item The Executive may grant exemptions to section \bref{service_req_exemption} on a case by case basis. In deciding whether to grant an exemption the Executive must consider:
  \begin{enumerate}
  \item The total number of service points accrued by the individual;
  \item The opportunities available to obtain service points;
  \item Any reasons given by the individual why they have not fulfilled the service requirements;
  \item Any other reasons given by the individual why they should be exempt from the service requirements.
  \end{enumerate}
\end{enumerate}
